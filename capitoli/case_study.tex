% !TEX encoding = UTF-8
% !TEX TS-program = pdflatex
% !TEX root = ../tesi.tex

%**************************************************************
\chapter{Study case: AWS EC2 resource generation with Pulumi}
\label{cap:case-study}
%**************************************************************

\intro{Generation of AWS EC2 resources with Pulumi to compare how various languages supported by Pulumi will differ in the infrastructure resources declaration}\\

%**************************************************************
\section{Amazon Web Services}
AWS is a wide collection of services with many different purposes and characteristics including compute, storage, databases, analytics, networking, mobile, developer tools, management tools, IoT, security, and enterprise applications: on-demand, available in seconds, with pay-as-you-go pricing.
Anyway, for the purpose of the thesis we'll focus only on the EC2 module.

\subsection{AWS's EC2 module}
EC2 provides scalable computing capacity in the Amazon Web Services (AWS) Cloud.
Amazon EC2 eliminates the need to invest in hardware up front, so that the development and deployment of the applications is faster.
Such a characteristics is a perfect fit for an IaC scenario.

\section{Case study infrastructure overview}
For the thesis, only few components of the vast EC2 module have been selected to create a working infrastructure.\\
The infrastructure

\subsection{Components of the infrastructure}

\subsubsection{VPC}

\subsubsection{Subnet}

\subsubsection{InternetGateway}

\subsubsection{RouteTable}

\section{Typescript implementation of the case study}

\section{Java implementation of the case study}

\section{My Scala implementation of the case study}

\subsection{Syntactic sugar}

\subsubsection{Syntactic sugar for the builders}

\subsubsection{Syntactic sugar for the constructors of the resources}

\subsubsection{Functor and Monads for the Output type}

\subsubsection{Usage of the syntactic sugar}

\subsection{Automatic code generation for the syntactic sugar}

\subsubsection{What Pulumi asks for the official support of a new Language}

\subsubsection{Raw automatic code generation with JavaParser}

\subsubsection{Usage of the generated code as a library}

\subsubsection{Possible future improvements for the code generation with Scalameta}
