% !TEX encoding = UTF-8
% !TEX TS-program = pdflatex
% !TEX root = ../tesi.tex

%**************************************************************
\chapter{Pulumi, an IaC platform}
\label{cap:introduction-pulumi}
%**************************************************************

\intro{An introduction to Pulumi}\\

%**************************************************************
\section{Introduction to Pulumi}
Pulumi is a cloud engineering platform that enables developers and infrastructure teams to build, deploy, and manage cloud-native applications and infrastructure across multiple cloud providers, including AWS, Azure, Google Cloud, and Kubernetes.\\
Pulumi provides a programming model that allows developers to use familiar languages, such as Python, JavaScript, TypeScript, Go, and (partially) Java to define their IaC and manage it as software.
In fact, it belongs to the second generation tools of the IaC.\\
As already mentioned in the \hyperref[cap:introduction-to-iac]{Introduction to Infrastructure as Code} chapter, such an approach, makes it easier to automate the deployment and management of infrastructure and applications, as well as to collaborate across teams and projects.\\
Pulumi offers a range of tools and features to simplify the development and management of cloud infrastructure, including version control, testing, continuous integration and delivery (CI/CD), monitoring, and security.
It also provides templates, examples, and libraries for common infrastructure patterns and services, such as containers, serverless functions, databases, and networking.\\
Overall, Pulumi aims to streamline the process of building and managing modern cloud-native applications and infrastructure, while providing a flexible and developer-friendly experience.

\subsection{The great advantages of Pulumi as a second generation IaC tool}
First of all, as mentioned in the \hyperref[sssec:second-wave]{Serverless Infrastructure as programming language code - the second wave} paragraph, all the functionalities that comes along a programming language are letting us achieve more robust and powerful solutions for our infrastructure, rather than what we could achieve with the expressive power of a markup language (like the ones used with \gls{Terraform}).
Being the focus of this thesis, we'll discuss more about such advantages in the \hyperref[cap:comparison-between-languages]{Comparison between the languages for Pulumi and the advantages of Scala Chapter}.\\
Furthermore, as aforementioned, Pulumi is a multi-cloud tool. Thanks to this we can rely on a single IaC tool for managing resources across different cloud platforms.\\
Moreover, Pulumi lets the user choose its favorite programming language, or the one that in its opinion is a best-fit for the need to be addressed.
In other words, such a choice can both reduce the requirements placed on the user's knowledge, since it can choose among many different programming languages, 
and at the same time offer different programming paradigms to choose from, so that for any need there is a programming language that is addressing such a need better than the others.\\
Finally, Pulumi comes with a range of integrated tools and features, such as automatic parallelism, drift detection, and stack references, making it easier to manage complex infrastructure and deployments.

