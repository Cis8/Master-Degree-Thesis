
%**************************************************************
% Acronimi
%**************************************************************
\renewcommand{\acronymname}{Acronyms and abbreviations}

\newacronym[description={\glslink{iacg}{Infrastructure as Code}}]
    {iac}{IaC}{Inftrastructure as Code}


%**************************************************************
% Glossario
%**************************************************************
%\renewcommand{\glossaryname}{Glossario}

%\newglossaryentry{iacg}
%{
%    name=\glslink{iac}{IaC},
%    text=Infrastructure as Code,
%    sort=infrastructure as code,
%    description={Infrastructure as Code (IaC) is the managing and provisioning of infrastructure through code instead of through manual processes}
%}
%
%\newglossaryentry{Pulumi}
%{
%    name=\glslink{Pulumi}{pulumi},
%    text=Pulumi,
%    sort=pulumi,
%    description={Pulumi is}
%}

\newglossaryentry{cloud computing}
{
    name=\glslink{cloud computing}{cloud computing},
    text=cloud computing,
    sort=cloud computing,
    description={Cloud computing is on-demand access, via the internet, to computing resources—applications, servers (physical servers and virtual servers), data storage, development tools, networking capabilities, and more—hosted at a remote data center managed by a cloud services provider}
}

\newglossaryentry{cloud service provider}
{
    name=\glslink{cloud service provider}{cloud service provider},
    text=cloud service provider,
    sort=cloud service provider,
    description={Cloud service providers are companies that establish public clouds, manage private clouds, or offer on-demand cloud computing components (also known as cloud computing services) like Infrastructure-as-a-Service (IaaS), Platform-as-a-Service (PaaS), and Software-as-a-Service(SaaS). Cloud services can reduce business process costs when compared to on-premise IT}
}

\newglossaryentry{mixin-based composition}
{
    name=\glslink{mixin-based composition}{mixin-based composition},
    text=mixin-based composition,
    sort=mixin-based composition,
    description={In scala, trait mixins means you can extend any number of traits with a class or abstract class. You can extend only traits or combination of traits and class or traits and abstract class,
    Traits are used to share interfaces and fields between classes. They are similar to Java 8’s interfaces }
}

\newglossaryentry{AWS}
{
    name=\glslink{AWS}{aws},
    text=AWS,
    sort=AWS,
    description={AWS stands for Amazon Web Services. It is a comprehensive, evolving cloud computing platform provided by Amazon that includes a mixture of infrastructure-as-a-service (IaaS), platform-as-a-service (PaaS) and packaged-software-as-a-service (SaaS) offerings. AWS services can offer an organization tools such as compute power, database storage and content delivery services}
}


\newglossaryentry{EC2}
{
    name=\glslink{EC2}{EC2},
    text=EC2,
    sort=EC2,
    description={EC2 stands for elastic computing 2. EC2 is a module of AWS that provides scalable computing capacity in the Amazon Web Services (AWS) Cloud }
}

\newglossaryentry{REST API}
{
    name=\glslink{REST API}{REST API},
    text=REST API,
    sort=REST API,
    description={A RESTful API is an architectural style for an application program interface (API) that uses HTTP requests to access and use data. That data can be used to GET, PUT, POST and DELETE data types, which refers to the reading, updating, creating and deleting of operations concerning resource}
}

\newglossaryentry{JavaParser}
{
    name=\glslink{JavaParser}{Javaparser},
    text=JavaParser,
    sort=JavaParser,
    description={JavaParser is a Java library that provides you with an Abstract Syntax Tree of your Java code. The AST structure then allows you to work with your Java code in an easy programmatic way}
}

\newglossaryentry{sequence comprehension}
{
    name=\glslink{sequence comprehension}{sequence comprehension},
    text=sequence comprehension,
    sort=sequence comprehension,
    description={In Scala, a sequence comprehension is a syntactic construct that allows you to create new sequences by transforming and filtering existing ones. It's similar to a for loop in other languages, but with a more concise and functional syntax }
}

\newglossaryentry{CDK}
{
    name=\glslink{CDK}{CDK},
    text=CDK,
    sort=CDK,
    description={The AWS Cloud Development Kit (AWS CDK) is an open-source software development framework for defining cloud infrastructure as code with modern programming languages and deploying it through AWS CloudFormation}
}

\newglossaryentry{Terraform}
{
    name=\glslink{Terraform}{Terraform},
    text=Terraform,
    sort=Terraform,
    description={Terraform is an open-source infrastructure as code (IaC) tool developed by HashiCorp. It lets define both cloud and on-prem resources in human-readable configuration files that you can version, reuse, and share. Developers to automate the creation, modification, and deletion of cloud resources in a scalable and efficient way}
}

\newglossaryentry{CloudFormation}
{
    name=\glslink{CloudFormation}{CloudFormation},
    text=CloudFormation,
    sort=CloudFormation,
    description={CloudFormation is an AWS managed service that provisions AWS cloud resources using templates written in JSON or YAML. With CloudFormation, configuration code is written in template files describing the desired cloud resources, then the code is uploaded to the CloudFormation service for evaluation and deployment.}
}

\newglossaryentry{Azure Resource Manager}
{
    name=\glslink{Azure Resource Manager}{Azure Resource Manager},
    text=Azure Resource Manager,
    sort=Azure Resource Manager,
    description={Azure Resource Manager is the deployment and management service for Azure. It provides a management layer that enables you to create, update, and delete resources in your Azure account
    }
}

\newglossaryentry{Azure}
{
    name=\glslink{Azure}{azure},
    text=Azure,
    sort=Azure,
    description={Microsoft Azure, formerly known as Windows Azure, is Microsoft's public cloud computing platform. It provides a broad range of cloud services, including compute, analytics, storage and networking}
}

\newglossaryentry{Cloud Deployment Manager}
{
    name=\glslink{Cloud Deployment Manager}{Cloud Deployment Manager},
    text=Cloud Deployment Manager,
    sort=Cloud Deployment Manager,
    description={Google Cloud Deployment Manager is an infrastructure deployment service that automates the creation and management of Google Cloud resources. Write flexible template and configuration files and use them to create deployments that have a variety of Google Cloud services, such as Cloud Storage, Compute Engine, and Cloud SQL, configured to work together}
}

\newglossaryentry{GCP}
{
    name=\glslink{GCP}{gcp},
    text=GCP,
    sort=GCP,
    description={Google Cloud Platform (GCP), offered by Google, is a suite of cloud computing services that runs on the same infrastructure that Google uses internally for its end-user products, such as Google Search, Gmail, Google Drive, and YouTube. Alongside a set of management tools, it provides a series of modular cloud services including computing, data storage, data analytics and machine learning}
}

\newglossaryentry{IaaS}
{
    name=\glslink{IaaS}{IaaS},
    text=IaaS,
    sort=IaaS,
    description={Infrastructure as a Service. It's a set of raw IT resources offered to the user by the cloud service provider. They can be used to virtualise an infrastructure, or for resource-intensive projects — i.e. machine learning, big data, hosting, etc}
}

\newglossaryentry{VPC}
{
    name=\glslink{VPC}{vpc},
    text=VPC,
    sort=VPC,
    description={A virtual private cloud (VPC) is a secure, isolated private cloud hosted within a public cloud. VPC customers can run code, store data, host websites, and do anything else they could do in an ordinary private cloud, but the private cloud is hosted remotely by a public cloud provider. (Not all private clouds are hosted in this fashion.) VPCs combine the scalability and convenience of public cloud computing with the data isolation of private cloud computing}
}

\newglossaryentry{subnet}
{
    name=\glslink{subnet}{Subnet},
    text=subnet,
    sort=subnet,
    description={In computer networking, a subnet is a smaller network within a larger network. It is created by dividing a larger network into smaller segments, with each segment having its own unique network address.\\
    A subnet is identified by its subnet mask, which is a 32-bit number used to divide an IP address into two parts - the network address and the host address. The subnet mask determines which bits in the IP address are used to identify the network and which bits are used to identify the hosts.\\
    Subnets are commonly used to improve network performance and security. By dividing a larger network into smaller subnets, network traffic can be reduced and more efficiently managed. Additionally, subnets can be used to isolate different parts of a network for security purposes, preventing unauthorized access to sensitive data or resources}
}

\newglossaryentry{internet gateway}
{
    name=\glslink{internet gateway}{internet gateway},
    text=internet gateway,
    sort=internet gateway,
    description={An internet gateway is a network component that serves as the entry and exit point between a private network, such as a Virtual Private Cloud (VPC) in the cloud, and the public internet.\\
    An internet gateway is responsible for translating the private IP addresses used within a private network to public IP addresses used on the internet, and vice versa. It enables instances within a VPC to communicate with the internet, and allows external users and resources to communicate with instances within the VPC.\\    
    An internet gateway typically includes two network interfaces - one that connects to the VPC, and another that connects to the public internet. It forwards traffic between the VPC and the internet, and routes traffic between the different subnets within the VPC}
}

\newglossaryentry{routing table}
{
    name=\glslink{routing table}{routing table},
    text=routing table,
    sort=routing table,
    description={In computer networking, a routing table, or routing information base (RIB), is a data table stored in a router or a network host that lists the routes to particular network destinations, and in some cases, metrics (distances) associated with those routes. The routing table contains information about the topology of the network immediately around it.\\
    In our case, the routing table has been used to bind the public subnets to the internet gateway, so that they can receive and send packets over the internet}
}

\newglossaryentry{abstract syntax tree}
{
    name=\glslink{abstract syntax tree}{AST},
    text=abstract syntax tree,
    sort=abstract syntax tree,
    description={In computer science, an abstract syntax tree (AST), or just syntax tree, is a tree representation of the abstract syntactic structure of text (often source code) written in a formal language. Each node of the tree denotes a construct occurring in the text}
}

\newglossaryentry{internal DSL}
{
    name=\glslink{internal DSL}{internal DSL},
    text=internal DSL,
    sort=internal DSL,
    description={DSL stands for Domain Specific Language. It’s a language which is built for a specific domain. A DSL is said to be internal if it is build basing on another programming language}
}