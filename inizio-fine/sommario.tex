% !TEX encoding = UTF-8
% !TEX TS-program = pdflatex
% !TEX root = ../tesi.tex

%**************************************************************
% Sommario
%**************************************************************
%\cleardoublepage
\phantomsection
\pdfbookmark{Sommario}{Sommario}
\begingroup
\let\clearpage\relax
\let\cleardoublepage\relax
\let\cleardoublepage\relax

\chapter*{Summary}

The document describes the work done from the graduating Cisotto Emanuele, in collaboration with Zoleo Andrea, from the Kynetics Srl company.
The objectives to achieve were many.\\
First, the study of the basics of the functioning of the Pulumi technology and the Scala language was made.\\
Second, a definition of a case study based on some \gls{AWS} \gls{EC2} resources was done. Hence, the respective implementation using Pulumi and the Typecript language has been done.\\
Then, after having identified the shortcomings of Pulumi with the Typescript language, a syntactic sugar for the Scala support for Pulumi has been manually coded.
Such a syntactic sugar for Scala is based on the Java APIs for Pulumi.\\
At that point, the same case study has been implemented with the newly created Scala APIs for Pulumi.\\
Thus, an inspection of the Pulumi's Java libraries was made with the \gls{JavaParser} library, to infer useful information about the structure of the APIs so that an automatic generation of the syntactic sugar code was possible.\\
Finally, the code to automatically generate all the Scala syntactic sugar code for all the AWS EC2 resources used in the case study was coded and used to generate such APIs.
The generated code was eventually exported as a library so that it could be used in any Pulumi Scala project.




%\vfill
%
%\selectlanguage{english}
%\pdfbookmark{Abstract}{Abstract}
%\chapter*{Abstract}
%
%\selectlanguage{italian}

\endgroup			

\vfill

