% !TEX encoding = UTF-8
% !TEX TS-program = pdflatex
% !TEX root = ../tesi.tex

%**************************************************************
% Sommario
%**************************************************************
%\cleardoublepage
\phantomsection
\pdfbookmark{Sommario}{Sommario}
\begingroup
\let\clearpage\relax
\let\cleardoublepage\relax
\let\cleardoublepage\relax

\chapter{Abstract}

\intro{Scala as a candidate to address the problem of the always more complex management of the cloud infrastructure}\\
\newline
In the last decades the infrastructure used by companies changed deeply.
Years ago they were used to have on premise bare-metal servers, running their own applications, while now they rent cloud servers or applications and host their services in a serverless fashion.\\
Such a modern approach granted enormous benefits to flexibility, but at the same time greatly increased the complexity to manage the configuration of such infrastructures.
The current situation requires companies to resort to advanced tools for managing the always more complex structure of the owned cloud resources.
The critical aspects of such tools are the range of cloud service providers (AWS, Azure, ecc.) supported and the programming language offered to work with such tools.\\
Among all the tools that started to appear, Pulumi is the one that results to be the most innovative.
With its multi-cloud and multiple general programming languages support, has all the features to address the high requirements imposed by the more and more complex cloud scenario.\\
This thesis shows why Scala represents a cool addition to the Pulumi's current pool of supported languages, and how a partial support of the language has been achieved in a very limited time.
\newline
\newline
\newline
\newline
\newline
\newline
%The document describes the work done from the graduating Cisotto Emanuele, in collaboration with Zoleo Andrea, from the Kynetics Srl company.
%The objectives to achieve were many.\\
%First, the study of the basics of the functioning of the Pulumi technology and the Scala language was made.\\
%Second, a definition of a case study based on some \gls{AWS} \gls{EC2} resources was done. Hence, the respective implementation using Pulumi and the Typecript language has been done.\\
%Then, after having identified the shortcomings of Pulumi with the TypeScript language, a \textit{syntactic sugar} for the Scala support for Pulumi has been manually coded.
%Such a \textit{syntactic sugar} for Scala is based on the Java APIs for Pulumi.\\
%At that point, the same case study has been implemented with the newly created Scala APIs for Pulumi.\\
%Thus, an inspection of the Pulumi's Java libraries was made with the \gls{JavaParser} library, to infer useful information about the structure of the APIs so that an automatic generation of the \textit{syntactic sugar} code was possible.\\
%Finally, the code to automatically generate all the Scala \textit{syntactic sugar} code for all the AWS EC2 resources used in the case study was coded and used to generate such APIs.
%The generated code was eventually exported as a library so that it could be used in any Pulumi Scala project.




%\vfill
%
%\selectlanguage{english}
%\pdfbookmark{Abstract}{Abstract}
%\chapter*{Abstract}
%
%\selectlanguage{italian}

\endgroup			

\vfill

