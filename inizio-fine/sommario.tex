% !TEX encoding = UTF-8
% !TEX TS-program = pdflatex
% !TEX root = ../tesi.tex

%**************************************************************
% Sommario
%**************************************************************
%\cleardoublepage
\phantomsection
\pdfbookmark{Sommario}{Sommario}
\begingroup
\let\clearpage\relax
\let\cleardoublepage\relax
\let\cleardoublepage\relax

\chapter*{Sommario}

Il presente documento descrive il lavoro svolto durante il periodo di \textit{stage}, della durata di circa trecentoventi ore, dal laureando Alessandro Rizzo presso l'azienda Infocert S.p.A.
Gli obiettivi da raggiungere erano molteplici.\\
In primo luogo era richiesto lo studio e la comprensione dei fondamenti della \textit{Chaos Engineering}.

In secondo luogo era richiesta l'implementazione di una versione rivisitata di un \textit{software} aziendale già esistente, MICO, in seguendo i principi dell'architettura a microservizi e reactive tramite il \textit{framework} Akka.
Tale \textit{framework} permette di utilizzare il modello ad attori per gestire il completamento di diversi task simultaneamente e in maniera asincrona.
In questo sviluppo andava applicato quanto appreso nella fase di studio per progettare e realizzare un'applicazione il più resiliente possibile.

Infine, una volta completato lo sviluppo, andavano applicati tutti i principi di \textit{Chaos Engineering} appresi durante la fase di studio per aumentare la fiducia nell'applicazione e per scoprire eventuali vulnerabilità non ancora considerate con lo scopo ultimo di aumentare la resilienza e l'affidabilità del prodotto.

%\vfill
%
%\selectlanguage{english}
%\pdfbookmark{Abstract}{Abstract}
%\chapter*{Abstract}
%
%\selectlanguage{italian}

\endgroup			

\vfill

